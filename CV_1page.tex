\documentclass{cv}

\usepackage[english]{babel} 
\usepackage[latin1]{inputenc}  %% les accents dans le fichier.tex
\usepackage[T1]{fontenc}       %% Pour la césure des mots accentués
\usepackage[paper=a4paper,textwidth=170mm]{geometry}
\usepackage[colorlinks=true, citecolor=blue,urlcolor=black]{hyperref}

\usepackage{amsmath, amssymb}

\usepackage{stackengine}
\usepackage{scalerel}
\usepackage{xcolor}
\usepackage{graphicx}
\newcommand\openbigstar[1][0.4]{%
  \scalerel*{%
    \stackinset{c}{-.125pt}{c}{}{\scalebox{#1}{\color{white}{$\bigstar$}}}{%
      $\bigstar$}%
  }{\bigstar}
}

\newcommand{\compresslist}{
	\setlength{\itemsep}{1pt}
	\setlength{\parskip}{0pt}
	\setlength{\parsep}{0pt}
}

\usepackage{textcomp}
\newcommand{\lieu}[1]{{#1}\ }
\newcommand{\activite}[1]{\textbf{#1}\ }
\newcommand{\comment}[1]{\textsl{#1}}
\geometry{top=1.5cm}

\usepackage{fancyhdr}
\pagestyle{fancy}
\fancyfoot[L]{} %CV Marine Lasbleis}
\fancyfoot[C]{}
\fancyfoot[R]{} %Page \thepage/5}
\renewcommand{\headrulewidth}{0pt}
\renewcommand{\footrulewidth}{1pt}

\author{Marine Lasbleis}

\begin{document}

\begin{minipage}{0.6\textwidth}
{\Large{\textbf{Marine Lasbleis, Ph.D.}

MSCA Post-doctoral Fellow}}
\end{minipage}
\hfill
\begin{minipage}{0.38\textwidth}
\begin{flushright}
Updated \today
\end{flushright}
\end{minipage}
\vspace{0.2cm}

\begin{minipage}{0.5\linewidth}
Laboratoire de Plan\'etologie et G\'eodynamique\\
(LPG)\\
Universit\'e de Nantes\\
Nationality: French.
\end{minipage}
\hfill
  \begin{minipage}{0.45\linewidth}
	Phone: +81 70 1572 5070\\%
	Email: \texttt{\href{mailto:marine.lasbleis@univ-nantes.fr}{marine.lasbleis@univ-nantes.fr}}\\
	Website:  \href{http://marinelasbleis.github.io}{http://marinelasbleis.github.io}\\
	Github account: \href{https://github.com/MarineLasbleis}{github.com/MarineLasbleis}\\

\end{minipage}

	%%%%%%%%%%%%%%%%%%
	% Bloc rubriques %
	%%%%%%%%%%%%%%%%%%

%\vspace{0.8cm}



%\begin{rubrique}{Current position}
%ELSI Research Scientist, at ELSI, Tokyo Institute of Technology, Tokyo, Japan.
%\end{rubrique}

%\begin{rubrique}{Research interests}
%Dynamics of planetary interiors, early Earth dynamics, inner core dynamics, magnetic field. 	\end{rubrique}
%\vspace{0.2cm}


\begin{rubriquetableau}[4cm]{Research and teaching positions}
% 2018 - now & \activite{ELSI Associate researcher.}\\
March 2019 - Now & \activite{MSCA-IF Post doctoral fellow,} LPG Nantes\\
& maternity leave: Dec. 2019 -- April 2020.\\
Sept. 2018 - Feb. 2019 & \activite{GeoPlanet Post doctoral researcher,} LPG Nantes\\
Apr. 2017 - Aug. 2018 & \activite{ELSI Research scientist.} ELSI, Tokyo, Japan\\
Apr. 2015 - March 2017
	& \activite{JSPS Post-doctoral Fellow FY2015} ELSI, Tokyo, Japan\\
Sept. 2011 -  Dec. 2014
& \activite{Teaching assistant} ENS de Lyon \\
& maternity leave: Jan. 2014 -- Mai 2014.
\end{rubriquetableau}


\begin{rubriquetableau}[2cm]{Education}
2014
& \activite{Ph.D.: Dynamics and evolution of the Earth's inner core}  
	\lieu{at Laboratoire de G\'{e}ologie de Lyon, France}. \comment{Ph.D. Defense: 4$^\mathrm{th}$ december 2014.} (First class honors)\\
 %2012    & \activite{Summer School} at the Cooperative Institute for Dynamic Earth Research
 %  (\href{http://www.deep-earth.org/}{\mbox{CIDER}})
 %  \textbf{"Deep Time: How did early Earth become our modern world?"} \comment{4 weeks}, UC Santa Barbara.\\
2010
	& \activite{M.S. in Physics} (Magna Cum Laude)
	\lieu{ENS de Lyon, France.}\\ 
2008	& \activite{B.S. in Earth Science} (Magna Cum Laude)
	\lieu{ENS de Lyon, France.}\\
%2007 & Competitive entrance at ENS de Lyon, France (ranking: 16, about 900 candidates).\\
\end{rubriquetableau}


\begin{rubrique}{Communications}%

	\vspace{-0.3cm}
	\begin{itemize}\compresslist
		\item \textbf{Peer review publications:} (student: underline)
	 \end{itemize}
	 \vspace{-0.3cm}


	 [2020] L. Noack, \textbf{M. Lasbleis}. \textit{Parameterisations of interior properties of rocky planets. 
	 An investigation of planets with Earth-like compositions but variable iron content} Astronomy\&Astrophysics, 2020, 638, A129.


	 [2020] \textbf{M. Lasbleis}, \underline{M. Kervazo}, G. Choblet. \textit{The fate of liquids trapped during the Earth's inner core growth}, Geophysical Research Letters, 2020, 47, e2019GL085654.



[2017]	 \textbf{M. Lasbleis}, L. Waszek, E. Day. \textit{GrowYourIC: a step towards a coherent model of seismic
	structure}, Geochemistry, Geophysics, Geosystems, 2017, 18 (11), 4016-4026.


[2015] \textbf{M. Lasbleis}, R. Deguen, S. Labrosse, P. Cardin. \textit{Earth's inner core dynamics
	induced by the Lorentz force}. Geophysical Journal International, 2015, 202 (1), 548-563. 

[2015] \textbf{M. Lasbleis}, R. Deguen. \textit{Building a regim diagram for the Earth's inner
  core}. Physics of the Earth and Planetary Interiors, 2015, 247, 80-93.  

[2013] \href{http://onlinelibrary.wiley.com/doi/10.1002/grl.50372/abstract}{Z. M. Geballe,
  \textbf{M. Lasbleis}, V. F. Cormier and E. A. Day.  \textit{Sharp hemisphere
    boundaries in a translating inner core} , Geophysical Research Letter, 2013,  40, 1719-1723}

[2011] \href{http://www.sciencedirect.com/science/article/pii/S0031920110001329}{J. Noir,
  M. A. Calkins,  \textbf{M. Lasbleis} and J. M. Aurnou.  \textit{Experimental study
    of librationally driven zonal flows in a straight cylinder} ,
  Physics of the  Earth and Planetary Interiors, 2010,  182, 98-106.}

 % \vspace{-0.3cm}
\begin{itemize}\compresslist
	\item \textbf{Non-peer review publications:}
\end{itemize}
\vspace{-0.3cm}

[2020] R. Deguen, \textbf{M. Lasbleis}. \textit{Fluid Dynamics of Earth's Core: Geodynamo, Inner Core Dynamics, Core Formation} In: Le Bars M., Lecoanet D. (eds) Fluid Mechanics of Planets and Stars. CISM International Centre for Mechanical Sciences (Courses and Lectures), vol 595. Springer, Cham, 2020.

[2017] E. Tasker, J. Tan, K. Heng, S. Kane, D. Spiegel, R. Brasser, A. Casey, S. Desch, C. Dorn, J. Hernlund, C. Houser, M. Laneuville, \textbf{M. Lasbleis}, A.-S. Libert, L. Noack, C. Unterborn \& J. Wicks. \textit{The language of exoplanet ranking metrics needs to change}. Nature Astronomy, 2017, 1 (2), 0042.

%\vspace{-0.3cm}
\begin{itemize}\compresslist
	\item \textbf{Submitted publications:}
\end{itemize}
\vspace{-0.3cm}

[Nature Geoscience] D. Frost, B. Romanowicz, \textbf{M. Lasbleis}, B. Chandler, \textit{Dynamic history of the inner core constrained by seismic anisotropy}.

[JGR Planet] \underline{I. Bonati}, \textbf{M. Lasbleis}, L. Noack, \textit{Structure and thermal evolution of exoplanetary cores}.



\begin{itemize} \compresslist
	\item 	\textbf{Invited talks at international conferences and workshops}
\end{itemize}
\vspace{-0.3cm}

SEDI 2020 (cancelled), the SEDI 2018 (Edmonton, Canada), the GRC Earth's Interior 217 (MA, USA).

%\begin{itemize}\compresslist%
%	\item \textbf{Softwares}
%\end{itemize}

%\textbf{M. Lasbleis}, 2017, MarineLasbleis/GrowYourIC 0.5: Zenodo, doi:10.5281/zenodo.3497530.

%	\textbf{Work in progress}
	
%	\textbf{M. Lasbleis}, J. Hernlund, S. Labrosse. \textit{Snow in the Earth's core}, in prep.

%	%\textbf{M. Lasbleis}, Q. Forquenot, R. Deguen. \textit{Inner core topography with stochastic approach}, in prep.

%	\textbf{M. Lasbleis}, M. Kervazo. \textit{Compaction during crystallisation of cores: implications for the Earth and rocky planets of the solar system}, in prep.
\end{rubrique}
\end{document}
%\newpage

\begin{rubrique}{Conferences and workshops}

\begin{itemize} \compresslist
	\item 	\textbf{Invited talks at international conferences and workshops}
\end{itemize}
	\vspace{-0.3cm}

% \hspace{-0.5cm}\textbf{Upcoming conferences and invitations}
% \begin{itemize}
% \end{itemize}

%\hspace{-0.5cm}\textbf{Past conferences and workshops}
	% \begin{itemize}\compresslist
	%\item[$\bigstar$] \textbf{M. Lasbleis}, M. Kervazo, \textit{Compaction during the evolution of planetary inner cores }, EPSC-DPS Meeting (Geneva, Switzerland), Setp. 2019.
	%\item \textbf{M. Lasbleis}, L. Noack, \textit{Can Massive Rocky Exoplanets Have a Solid Inner Core?}, AGU (American Geophysical Union) Fall Meeting (Washington DC, USA), December 2018.
	%\item M. Kervazo, \textbf{M. Lasbleis}, \textit{Compaction of a mushy inner core : the fate of liquid trapped by fast growth}, AGU (American Geophysical Union) Fall Meeting (Washington DC, USA), December 2018.
	 SEDI (Study of the Earth Deep Interior) meeting, July 2018 (Edmonton, Canada)
	%\item[$\bigstar$] \textbf{M. Lasbleis}, R. Deguen, Q. Forquenot, \textit{Topography at the inner core boundary}, PICO Presentation EGU (European Geophysical Union) meeting (Vienna, Austria), April 2018.
	%\item[$\bigstar$] \textbf{M. Lasbleis}, R. Deguen, Q. Forquenot, \textit{Topography at the inner core boundary}, AGU Fall Meeting (New Orleans, USA), December 2017.
	%\item E. Day, \textbf{M. Lasbleis}, L. Waszek,  \textit{GrowYourIC: an open access Python code to facilitate comparison between kinematic models of inner core evolution and seismic observations}, AGU Fall Meeting (New Orleans, USA), December 2017.
	%\item \textbf{M. Lasbleis}, L. Waszek, E. Day,  \textit{A first step to compare geodynamical models and seismic observations of the inner core}, IAG-IASPEI meeting (Kobe, Japan), August 2017.
	%\item Q. Forquenot, \textbf{M. Lasbleis}, \textit{Inner core topography}, Crust to Core Workshop (Ikoinoie, Japan), July 2017. 
	%\item \textbf{M. Lasbleis}, R. Brasser, \textit{Can we estimate an upper bound for the magnetic field of rocky planets?} ACCRETE Meeting (Nice, France), May 2017
	
	Gordon Research Conference on the Earth Interior  (South Hadley, MA, USA), May 2017. 
	%\item \textbf{M. Lasbleis}, R. Brasser, \textit{Can we estimate an upper bound for the magnetic field of rocky planets?} Astrobiology Science Conference (Mesa, AZ, USA), April 2017. 
	%\item	\textbf{M. Lasbleis}, L. Waszek, E. Day,  \textit{A first step to compare geodynamical models and seismic observations of the inner core}, AGU (American Geophysical Union) Fall Meeting (San Francisco, CA, USA), December 2016.
	%\item[$\openbigstar$]   \textbf{M. Lasbleis}, \textit{Deep Earth from a laptop}, European Research Day (Tokyo, Japan), November 2016.
	%\item \textbf{M. Lasbleis}, J. Hernlund, S. Labrosse, \textit{Solid iron snow in the F-layer}, SEDI (Study of the Earth Deep Interior) meeting, July 2016.
	%\item M. Laneuville, \textbf{M. Lasbleis}, G. Helffrich, \textit{Evolution of an Initially Stratified Liquid Core on Mars and Dynamo activity}, SEDI (Study of the Earth Deep Interior) meeting, July 2016.
	%\item[$\bigstar$] \textbf{M. Lasbleis}, J. Hernlund, S. Labrosse, \textit{Snow in the Earth's core}, Goldschmidt Conference (Yokohama, Japan), June 2016.
	%\item[$\bigstar$] \textbf{M. Lasbleis}, J. Hernlund, S. Labrosse, \textit{Snow Model for the F-Layer}, AGU (American Geophysical Union) Fall Meeting (San Francisco, CA, USA), December 2015.
	%\item L. Waszek, \textbf{M. Lasbleis}, E. Day, D. Al-Attar, Z.M. Geballe \textit{Super-rotation, Translation and Growth of the Inner Core: A Step Towards a Coherent Model of Seismic Structures}, AGU (American Geophysical Union) Fall Meeting (San Francisco, CA, USA), December 2015.
	
	Workshop on the Earth's mantle and core (Matsuyama, Japan) November 2015.
	%\item \textbf{M. Lasbleis}, S. Labrosse, R. Deguen, \textit{Building a regime diagram for the Earth's inner core}, AGU (American Geophysical Union) Fall Meeting (San Francisco, CA, USA), December 2014.
	%\item \textbf{M. Lasbleis}, R. Deguen, S. Labrosse, P. Cardin, \textit{Dynamics induced by the Lorentz force in the growing inner core}, SEDI (Study of the Earth Deep Interior) Meeting (Kamakura, Japan) August 2014.
	
	Workshop on transport properties in the Earth's core (Kawaguchiko, Japan), November 2013.
	
	 Inner core anisotropy workshop (Grenoble, France), November 2013.

\vspace{-0.1cm}

\begin{itemize}\compresslist
	\item \textbf{Contributed talks at international conferences and workshops:}
\end{itemize}
\vspace{-0.3cm}

EPSC-DPS 2019, AGU 2017, AGU 2015, AGU 2011.

%\begin{itemize}
%	\item[$\bigstar$] \textbf{M. Lasbleis}, M. Kervazo,  EPSC-DPS Meeting (Geneva, Switzerland), Setp. 2019.
%	\item[$\bigstar$] \textbf{M. Lasbleis}, R. Deguen, Q. Forquenot,  AGU Fall Meeting (New Orleans, USA), December 2017.
	% \item[$\bigstar$]   \textbf{M. Lasbleis}, \textit{Deep Earth from a laptop}, European Research Day (Tokyo, Japan), November 2016.
%	\item[$\bigstar$] \textbf{M. Lasbleis}, J. Hernlund, S. Labrosse,  AGU  Fall Meeting (SF, CA, USA), December 2015.
	%\item L. Waszek, \textbf{M. Lasbleis}, E. Day, D. Al-Attar, Z.M. Geballe \textit{Super-rotation, Translation and Growth of the Inner Core: A Step Towards a Coherent Model of Seismic Structures}, AGU (American Geophysical Union) Fall Meeting (San Francisco, CA, USA), December 2015.
%	\item[$\bigstar$] \textbf{M. Lasbleis}, S. Labrosse, J. Hernlund, AGU Fall Meeting (SF, CA, USA), December 2011.
%\end{itemize}

\vspace{0.2cm}
And more than 15 contributions as presenter of a poster at an international conference. 
\end{rubrique}
\end{document}


%\end{rubrique}
%\newpage
%\begin{rubrique}{Conferences and workshops -- Talks $\bigstar$ and invited talks $\openbigstar$.}%
%	 \begin{itemize}%
%	\item[$\openbigstar$] \textbf{M. Lasbleis}, R. Deguen, S. Labrosse, P. Cardin, \textit{Dynamics induced by the Lorentz force in the growing inner core} Inner core anisotropy workshop (Grenoble, France), November 2013. 
%	\item \textbf{M. Lasbleis}, R. Deguen, S. Labrosse, P. Cardin, C. Juan, \textit{Dynamics induced by the Lorentz force in the growing inner core}, Gordon Research Conference and Gordon Research Seminar on the Earth Interior (South Hadley, MA, USA) June 2013.
%	\item \textbf{M. Lasbleis}, S. Labrosse, J. Hernlund, \textit{Growth of the inner core by snowfall}, AGU (American Geophysical Union) Fall Meeting (San Francisco, CA, USA) December 2012.
%	%\item E. Day, V. Cormier, Z. Geballe, \textbf{M. Lasbleis}, M. Youssof, H. Yue, \textit{Investigating the translation of Earth's inner core}, AGU (American Geophysical Union) Fall Meeting (San Francisco, CA, USA) December 2012.
%		\item \textbf{M. Lasbleis}, B. Journaux, C. Perge, A. Marguerite, H. Dore, B. Bourget, \textit{A planet in your garage : Simple and memorable experiments to explain complex fluid dynamic phenomenons to non-scientist} -- Education session, AGU (American Geophysical Union) Fall Meeting (San Francisco, CA, USA) December 2012.
%	\item \textbf{M. Lasbleis}, S. Labrosse, J. Hernlund, \textit{Can the F-layer be explained by a slurry layer?}, SEDI (Study of the Earth Deep Interior) Meeting (Leeds, UK) July 2012.
%	\item[$\bigstar$] \textbf{M. Lasbleis}, S. Labrosse, J. Hernlund, \textit{Feedbacks Between Inner Core Freezing, Melting, and Evolution of the F-Layer}, AGU (American Geophysical Union) Fall Meeting (San Francisco, CA, USA), December 2011.
%	\end{itemize}

\textbf{Invited seminars:}
	\begin{itemize}
	\item UMET, Universit\'e de Lille (May 2019)
	\item IMPMC, Sorbonne Universit\'e (January 2019)
	\item Research School of Earth Sciences, Australian National University (February 2018)
	\item Laboratoire de Plan\'etologie de Nantes, France (March 2015, May 2016).
	\item Institut des Sciences de la Terre, Grenoble, France (March 2015, April 2016).
	\item Institut de Recherches en Astrophysique et Plan\'etologie, Toulouse, France (April 2016).
	\end{itemize}
\end{rubrique}


\begin{rubriquetableau}[1cm]{Fundings/scholarships/awards} 
	2019 & Member of the ISSI (International Space Science Institute) International Working Group \textit{Non-Equilibrium Iron Snow}\\
	2018 & Marie Sklodowska-Curie actions (MSCA) Individual Fellowship \\
	&(185k EUR for 2 years, starting March 2019)\\
	2018 & Grant-in-Aid for Scientific
	Research (Kakenhi -- young researcher)$\sim$ 17k EUR for 2 years (PI)\\
	2018 & Short Term Stay Mission COST Action TD 1308 ORIGINS, 900 EUR.\\
	2017 & Earth-Life Science Institute Incentive Award. \\
	2016 & Earth-Life Science Institute Director's fund $\sim$ 7k EUR.(PI)\\
	2015 & Grant-in-Aid for Scientific
	Research (Kakenhi -- JSPS postdoctoral fellow). \\
	& $\sim$ 17k EUR for 2 years. (Fiscal years 2015 and 2016) \\
	2015 & Earth-Life Science Institute Director's fund $\sim$ 6k EUR. (PI)\\
	2015 & Japan Society for the Promotion of Science Post Doctoral Fellowship (2 years). $\sim$ 70kEUR \\
	2012 & CIDER funding for research group. 5k EUR. (co-I)\\
	2011 & \'{E}cole Normale Sup\'{e}rieure de Lyon grant for student research groups. 4.5k EUR. (PI)\\ 
	2007 & Scholarship: full study and living allowance grant for graduate studies at \'Ecole Normale Sup\'{e}rieure de Lyon: $\sim$ 60k EUR. 
\end{rubriquetableau}



\begin{rubrique}{Teaching experience and teaching administration}
	2019 \activite{5-hour tutorial on numerical modeling in Dynamique des Int\'erieurs Plan\'etaires,} M1 STU Universit\'e de Nantes.\\

	2018 \activite{13-hour tutorial on numerical modeling}
	for the 2nd GeoPlaNet Thematic School Fluid-Rock Interactions
	in the Solar System 12th-16th November 2018, Nantes (France).\\

	2018 \activite{Co-organization of the 2018 ELSI/EON Winter School for graduate and post doctoral students (34 students, from 4 different continents),} with responsabilities for the curriculum and technical organisation of the lectures: initiation to UNIX and Python, planetary atmospheres, exoplanets, mantle convection. Lectures on initiation to UNIX and Python.\\

	2015 \activite{Highschool class} \lieu{Chiba, Japan.} 3 hours on Introduction to the Earth's core.\\

2011 - 2014 
	\activite{Teaching assistant} \lieu{at the Department of Earth Science, \'Ecole Normale Sup\'erieure (ENS) de Lyon, France.} Total: 190 hours over 3 years.
	In bold are the lectures I was responsible for in term of curriculum and organisation.
	\vspace{0.5cm}

	\noindent
	\begin{tabular}{llll}
		\hline
		B.S. 2nd y. & Introduction to geophysics & Spring 2012 & 20hours (tutorials)\\
		\hline
		B.S. 3rd y.& \textbf{Computer Science for geosciences} & Fall 2012, 2013 & 2x20h (lectures+tut.)\\
		& \textbf{Mathematics} & Fall 2011, 2012, 2013 & 3x20h (lectures+tut.)\\
		& Thermodynamics of natural systems & Fall 2011, 2012, 2013 & 3x14h (tutorials)\\
		& Geophysics (elasticity and magnetism)& Spring 2012, 2013 & 2x25h (tutorials)\\
		& Mentoring & Fall 2012, 2013 & 2x20h\\
		\hline
		M.S. 1st y. & Geophysics (heat budget of planets) & Fall 2012, 2013 & 2x6h (tutorials)\\
		\hline
	\end{tabular}
\end{rubrique}

\begin{rubriquetableau}[2cm]{Mentoring activities}
	2019 & \textbf{Marine Gelin}, undergrad (Licence 3) student internship, Inner core dynamics with CIG-ASPECT.\\
	2018 & \textbf{Irene Bonati}, graduate student at ELSI (principal advisor: John Hernlund). Project: magnetic field of exoplanets.\\
	2017 & \textbf{Mathilde Kervazo}, graduate (Master 1 and M2) student internship (3 months and 1 month) at ELSI: Inner core dynamics with CIG-ASPECT and inner core compaction. Research presented at AGU 2018, EPSC 2019 and under review for publication.\\
	2017 & \textbf{Quiterie Forquenot}, graduate (Master 2) student internship (5 months) at ELSI: Inner core topography. Research presented at the Crust-to-core workshop (July 30-Aug. 1, Ikoinoie, Japan)\\
	2016 & \textbf{Maude Geissmann}, graduate (Master 1) student internship,  at ELSI, co-advised with John Hernlund. Research presented at the SEDI meeting 2016.\\
	2013 & \textbf{Christie Juan}, undergrad (Licence 3) student internship (1 month).\\
\end{rubriquetableau}


\begin{rubriquetableau}[2cm]{Scientific Outreach}
	2019 & Hands-on experiments at the Fete de la Science 2019.\\
	2017 & Talk at Nerd Nite Tokyo, October 20.\\
	2017 & Talk at the ELSIOrigins1, September 2017. Public outreach event at ELSI. \\
	2016 & Talk on "Research and life in Japan", at the {Orientation for JSPS Post Doctoral Fellows.}\\ 
	2016 & Organisation of the {"Earth-Life Science Institute Youchien"}, classes held at the institute. Lecture on "fluid dynamics for deep Earth".\\
	2016 & ELSI's {Ask me Anything}: open discussion with the public at the Tokyo Tech Institute of Technology.\\
	2012 - 2014 & Participation to the annual science fair \textit{F\^ete de la Science} (4 days, responsible for the organisation and presentation of an activity for the 4 days), ENS de Lyon.\\ 
	2011 - 2013 
&{Creation and management of a student group: 
    First approach of planetary fluid dynamics}
\lieu{ENS de Lyon.}. Budget: 4.5k euros for two years. \\
	%& Tsunami generation and propagation in a 2-meter tank, rotating fluid on a small turntable, introduction to viscosity and viscous flows with kitchen liquids, first approach to magnetic field with ferrofluids and magnets.
	%&Award for a lecture given to 5th grade students on tsunamis and risks ("Prix La Main \`a la p\^ate 2013").
\end{rubriquetableau}

% ================================
%\newpage


\begin{rubriquetableau}[2cm]{Community service and other services}
	2019 & Primary convener of AGU 2019 Fall Meeting session DI13A-DI14A. \\
	2019 & Co-convener of Goldschmidt 2019 session Beyond Earth and Earth-like: The Compositional Diversity and Evolution of Exoplanets\\
	2018 & Primary convener of AGU 2018 Fall Meeting session DI14.\\
	2018 - now & Contact information between ELSI and the LPG for the consortium GeoPlanet.\\
	2018 & Co-organizer of the Earth Life Science Winter School, at ELSI, Jan-Feb 2018.\\
	2017 - now & Community ambassador for EarthArXiV (an open source community governed preprint system for the Earth sciences)\\
	2017 & Organizer of the Power Hour during Gordon Research Conference Interior of the Earth (South Hadley, MA, USA)\\
	2016 - 2017 
	&{Member of the Scientific Organizing Committee} for the 5th ELSI International Symposium (11-13 January 2017)\\ 
	2016 - 2017 & 	Organizer of the Young Researchers' Day during the 5th ELSI Symposium (10 January 2017. $\sim$ 60 grad students and post docs). \\
	2016 & {Convener} of AGU 2016 Fall meeting SEDI session.\\
	2014 - now & {Outstanding student paper award judge}, AGU fall meeting.\\
	2012 - 2014 & Grad student representative at the {Laboratoire de G\'{e}ologie de Lyon board of directors}.\\
	2011 &{Organization of the graduate student Day 2011:} short scientific conferences
held by the grad students, \lieu{Laboratoire de G\'{e}ologie de
Lyon.} Budget: 2.5k euros.\\
2008 - 2011 & {Organization of multidisciplinary scientific conferences} \comment{(
  C\'edric Villani, Thomas Peacock, Eric Brun, 
  Louis Schweitzer, \textit{etc.})} Budget: 2k euros per year.\\
	2008 - 2010 & Student representative at the {Board of Directors of the ENS de Lyon.}\\
& \\
& \textbf{Peer-review for:} Icarus, Progress in Earth and Planetary Science, Nature Advances.\\
& \textbf{Academic membership:} American Geophysical Union (2011 -- today), European Geophysical Union (2017 -- today), EANA (2018 -- today), Europlanet (2019 -- today)\\
\end{rubriquetableau}



%\newpage


\end{document}
